\documentclass[a4paper,9pt]{article}
\usepackage[utf8]{inputenc}
\usepackage[russian]{babel}
\usepackage{amsmath}
\usepackage{graphicx}
\usepackage{multicol}
\usepackage{geometry}
\usepackage{array}
\usepackage{titlesec}

\geometry{top=2cm,bottom=3cm,left=1.5cm,right=1.5cm}

% Уменьшение отступа между колонками
\setlength{\columnsep}{0.5cm}

\begin{document}

\begin{center}
    {\LARGE \textbf{XXXVII Международная математическая олимпиада}}
\end{center}

\begin{multicols}{3}

С 5 по 17 июля в Бомбее (Индия) состоялась XXXVII Международная математическая олимпиада, ставшая рекордной по числу участников: 426 школьников из 75 стран мира. Ещё 3 страны прислали своих официальных наблюдателей, а это значит, что в следующем году команды этих стран станут полноправными участниками состязаний.

В команду России в этом году вошли:

\textit{Николай Дуров} — девятиклассник ФМЛ 239 из Санкт-Петербурга, 

\textit{Вероника Есаулова} — выпускница ФМЛ 239 из Санкт-Петербурга, 

\textit{Юрий Макарычев} — выпускник с.ш. 57 из Москвы,  

\textit{Сергей Норин} — выпускник ФМЛ 239 из Санкт-Петербурга,  

\textit{Елена Рудо} — выпускница ФМЛ 239 из Санкт-Петербурга,  

\textit{Константин Салихов} — выпускник СУНЦ МГУ из Казани.

Запасным участником был выпускник ФМЛ 239 \textit{Дмитрий Запорожец}. Все участники команды получили право поступления в избранные ими вузы без вступительных экзаменов.

Команда была сформирована по итогам заключительного этапа Российской олимпиады, а с целью проверки её боеготовности в Санкт-Петербурге с 14 по 28 июня были проведены учебно-тренировочные сборы.

\columnbreak

Олимпиада проводилась, как обычно, в два дня, в течение которых участникам предлагалось решить по три задачи (на это отводилось 4,5 часа в каждый из этих дней), полное решение каждой задачи оценивалось семью баллами. Задачи олимпиады включены в «Задачник «Кванта» этого номера, за исключением задачи 3 первого дня.
  
Пусть \(S\) — множество неотрицательных целых чисел. Найдите все функции \(f\), определенные на \(S\) и принимающие свои значения в \(S\), такие, что\[f(m + f(n)) = f(f(m)) + f(n)\]
для всех \(m, n\) из \(S\).

\hfill (Румыния)

Олимпиада этого года оказалась одной из самых трудных за последние несколько лет. Золотая медаль присуждалась участникам, набравшим 28 и более очков из 42 возможных, серебряная - набравшим от 20 до 27 очков, бронзовая - набравшим от 12 до 19 очков. Не слишком высокие результаты объясняются ещё и тяжёлыми климатическими условиями, в которых проходила олимпиада (сильная жара и почти 100-процентная влажность).

Результаты команды России приведены в таблице 1, а таблица 2 содержит результаты выступления стран в неофициальном командном зачёте (напомним, что, согласно положению, олимпиада представляет собой личное первенство).

Хочется отметить успешное выступление Сергея Норина, получившего золотую медаль в третий раз подряд, а также результат самого молодого участника нашей команды Николая Дурова.

В третий раз получил золотую медаль \textit{Юлий Санников} (Украина). Золотыми медалями были награждены \textit{Сергей Чих} (Белоруссия), \textit{Давид Чхаидзе} (Грузия) и \textit{Лев Буховский} (Израиль) - воспитанник ФМЛ 239 Санкт-Петербурга.

Следующая международная математическая олимпиада состоится в июле 1997 года в городе Мар-дель-Плата (Аргентина).

\end{multicols}

\begin{table}[h!]
\caption{Результаты участников команды России}
\centering
\begin{tabular}{|c|c|c|c|c|c|c|c|}
\hline
& \multicolumn{7}{c|}{Задача}\\ \hline
Участник             & 1 & 2 & 3 & 4 & 5 & 6 & \sum \\ \hline
Николай Дуров        & 7 & 7 & 7 & 7 & 7 & 2 & 37   \\ \hline
Вероника Есаулова    & 4 & 7 & 3 & 7 & 1 & 3 & 25   \\ \hline
Юрий Макарычев       & 6 & 7 & 5 & 1 & 0 & 0 & 19   \\ \hline
Сергей Норин         & 7 & 7 & 7 & 7 & 7 & 1 & 36   \\ \hline
Елена Рудо           & 2 & 1 & 6 & 7 & 0 & 7 & 23   \\ \hline
Константин Салихов   & 2 & 7 & 5 & 7 & 1 & 0 & 22   \\ \hline
\end{tabular}
\end{table}
\newpage
\begin{table}[h!]
\caption{Неофициальный командный зачёт по результатам стран}
\centering
\begin{tabular}{|c|c|c|}
\hline
\textbf{Страна}        & \textbf{Очки} & \textbf{Медали (з + с + б)} \\ \hline
1. Румыния             & 187           & 4 + 2 + 0                   \\ \hline
2. США                 & 185           & 4 + 2 + 0                   \\ \hline
3. Венгрия             & 167           & 3 + 2 + 1                   \\ \hline
4. Россия              & 167           & 3 + 2 + 1                   \\ \hline
5. Великобритания      & 162           & 2 + 3 + 1                   \\ \hline
6. Китай               & 160           & 3 + 3 + 0                   \\ \hline
7. Вьетнам             & 155           & 3 + 1 + 1                   \\ \hline
8. Корея               & 151           & 2 + 3 + 0                   \\ \hline
9. Иран                & 143           & 1 + 4 + 1                   \\ \hline
10. Германия           & 137           & 3 + 1 + 1                   \\ \hline
11-12. Япония          & 136           & 1 + 3 + 1                   \\ \hline
11-12. Болгария        & 136           & 1 + 4 + 1                   \\ \hline
13. Польша             & 122           & 0 + 3 + 3                   \\ \hline
\end{tabular}
\quad
\begin{tabular}{c|c|c}
\hline
\textbf{Страна}        & \textbf{Очки} & \textbf{Медали (з + с + б)} \\ \hline
14. Индия              & 118           & 1 + 3 + 1                   \\ \hline
15. Израиль            & 114           & 1 + 2 + 2                   \\ \hline
...18. Украина         & 105           & 1 + 0 + 5                   \\ \hline
...21. Белоруссия       & 99           & 1 + 1 + 2                   \\ \hline
...30. Грузия           & 78           & 1 + 0 + 2                   \\ \hline
...32. Литва            & 68           & 0 + 1 + 2                   \\ \hline
33. Латвия              & 66           & 0 + 0 + 3                   \\ \hline
34-35. Армения          & 63           & 0 + 0 + 1                   \\ \hline
...41. Молдавия         & 55           & 0 + 0 + 2                   \\ \hline
\end{tabular}
\end{table}

\begin{center}
    {\LARGE \textbf{Задачи олимпиады}}
\end{center}

\begin{multicols}{3}

\textbf{6 класс}

\begin{enumerate}
    \item Выразите числа 5, 26, 30 и 55, используя четыре цифры 5, знаки арифметических действий и скобки. Например: \(3=(5+5+5):5\).
    \item Найдите наименьшее натуральное число, которое при делении на 5 дает остаток 4, при делении на 7 - остаток 6, при делении на 9 - остаток 8.
    \item В сказочном озере плавает сказочная лилия. Эта лилия за сутки вдвое увеличивает свои размеры и полностью заполняет озеро за 137 суток. За какое время заполнят озеро две сказочные лилии?
    \item Вова, Петя и Коля сварили уху и съели её поровну. Для ухи Вова дал 5 рыб, Петя - 3 рыбы. Коля рыбу не поймал и отдал за уху 2400 рублей. Как Вова и Петя должны разделить между собой эти деньги, чтобы дележ оказался справедливым?
    \item Лист бумаги разрезали на 4 части. Затем некоторые или все из этих частей опять разрезали на 4 части и т.д. Можно ли в результате получить 50 листочков бумаги любого размера?
    \item Найдите сумму 100 дробей: \\[0.5em]
    \(\frac{1}{1*2}+\frac{1}{2*3}+\frac{1}{3*4}+...\)
    \begin{center}
        \(...+\frac{1}{98*99}+\frac{1}{99*100}+\frac{1}{100*101}\).
    \end{center}
\end{enumerate}

\textbf{7 класс}

\begin{enumerate}
    \item В комнате стоят стулья и табуретки. У каждой табуретки 3 ножки, у каждого стула - 4 ножки. Когда на всех стульях сидят люди, в комнате 39 «ног». Сколько стульев и табуреток в комнате?
    \item Постройте график функции
    \begin{center}
        \(\frac{x+3y}{x+2y+1}=1\)
    \end{center}
    \item Разложите на множители выражение \(x^4+x^2+1\).
    \item У Пети есть 44 монеты и 10 карманов. Сможет ли он разложить свои монеты по карманам так, чтобы количество монет было различным? Ответ объясните.
    \item Найдите натуральные \(x\), которые являются решениями уравнения
    \begin{center}
        \((x+3)(x+4)(x+5)=4735\).
    \end{center}
    \item Докажите, что сумма длин медиан треугольника больше 3/4 его периметра.
\end{enumerate}

\textbf{8 класс}

\begin{enumerate}
    \item Решите систему уравнений
    \begin{center}
        \(\begin{cases}
            \frac{4}{x+y-1}-\frac{5}{2x-y+3}+\frac{5}{2}=0,\\[0.5em]
            \frac{3}{x+y-1}+\frac{1}{2x-y+3}+\frac{7}{5}=0
        \end{cases}\)
    \end{center}
    \item Постройте график функции \\[0.5em]
    \(y=-2x+\frac{1}{\sqrt{x+3}}+3-\)\\[0.5em]
    \(-\frac{1}{\sqrt{x+3}}-\frac{1}{x-1}+\frac{1}{\sqrt{4-x}}+\)\\[0.5em]
    \(+\frac{1}{x-1}-\frac{1}{\sqrt{4-x}}\)
    \item Является ли число
    \begin{center}
        \(\sqrt{3+2\sqrt{2}}+\sqrt{6-4\sqrt{2}}\) \\[0.5em]
    \end{center}
    рациональным?
    \item Есть только два двузначных числа, каждое из которых равно неполному квадрату разности своих цифр. Найдите эти числа, если одно число на 11 больше другого числа.
    \item Найдите сумму 99 дробей: \\[0.5 em]
    \(\frac{1}{\sqrt{2}+\sqrt{1}}+\frac{1}{\sqrt{3}+\sqrt{2}}+\frac{1}{\sqrt{4}+\sqrt{3}}+...\)
    \begin{center}
        \(...+\frac{1}{\sqrt{99}+\sqrt{98}}+\frac{1}{\sqrt{100}+\sqrt{99}}\).
    \end{center}
    \item Даны отрезки длиной \(a\) и \(b\). Постройте с помощью циркуля и линейки отрезок длиной
    \begin{center}
        \(\sqrt{a^2-2ab+2b^2}\)
    \end{center}
\end{enumerate}

\textbf{9 класс}

\begin{enumerate}
    \item Найдите все целые числа \(m\) и \(n\), удовлетворяющие равенству
    \begin{center}
        \(m(m-2n)=4n^2\),
    \end{center}
    и докажите, что других таких целых чисел не существует.
    \item Докажите, что при любых действительных числах \(x\) и \(y\) имеет место неравенство
    \begin{center}
        \(x^2+2xy+3y^2+2x+6y+4\geq1\).
    \end{center}
    \item Постройте график уравнения
    \begin{center}
        \(|y|+\frac{1}{|y|}=|x|+\frac{1}{|x|}\).
    \end{center}
    \item Докажите, что число
    \begin{center}
        \(\underset{\text{2n цифр}}{\text{11...1}}-\underset{\text{n цифр}}{\text{222...2}}\)
    \end{center}
    является квадратом натурального числа (например: \(11-2=9=3^2\), \(1111-22=1089=33^2\) и т.д.).
    \item Решите уравнение
    \begin{center}
        \(x\frac{50-x}{x+1}(x+\frac{50-x}{x+1})=576\).
    \end{center}
    \item Периметр треугольника \(ABC\) равен \(a\). Прямая \(A_1C_1\), параллельная основанию \(AC\), отсекает от треугольника \(ABC\) треугольник \(A_1BC_1\), периметр которого равен \(b\). Найдите длину основания \(AC\), если известно, что в трапецию \(AA_1C_1C\) можно вписать окружность.
\end{enumerate}

\textbf{10 класс}

\begin{enumerate}
    \item Докажите, что из равенства \(a^2+b^2+c^2=ab+bc+ac\), где \(a,b,c\) - действительные числа, следует, что \(a=b=c\).
    \item Пусть \(a<b<c\). Докажите, что уравнение
    \begin{center}
        \(\frac{1}{x-a}+\frac{1}{x-b}+\frac{1}{x-c}=0\)
    \end{center}
    имеет ровно два корня \(x_1\) и \(x_2\), удовлетворяющие неравенствам
    \begin{center}
        \(a<x_1<b<x_2<c\)
    \end{center}
    \item Решите уравнение
    \begin{center}
        \(x^3+1=2\sqrt[3]{2x-1}\).
    \end{center}
    \item В треугольнике взята произвольная точка, через которую проведены прямые, параллельные сторонам треугольника. Площади трёх полученных при этом треугольников равны \(S_1\), \(S_2\), \(S_3\). Найдите площадь исходного треугольника.
    \item Упростите выражение \\[0.5em]
    \(\tan\alpha+2\tan\2\alpha+4\tan4\alpha+\) \\
    \(+8\tan8\alpha+16\tan16\alpha+\) \\
    \(+32\tan32\alpha\).
    \item Постройте график функции
    \(y=\log_{1/2}(x-\frac{1}{2})+\)
    \begin{center}
        \(+\log_{2}\sqrt{4x^2-4x+1}+0.5^{1-2x}\).
    \end{center}
\end{enumerate}
\end{multicols}
\end{document}
